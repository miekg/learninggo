{.exercise difficulty="1"}
### Functions that return functions

1. Write a function that returns a function that performs a $$+2$$ on integers. Name the function `plusTwo`.
    You should then be able do the following:

        p := plusTwo()
        fmt.Printf("%v\n", p(2))

    Which should print 4. See (#sec:callbacks).

2. Generalize the function from above and and create a `plusX(x)` which returns functions that add `x` to an integer.


{.answer}
### Answer
1. 

{callout="//"}
    func main() {
            p2 := plusTwo()
            fmt.Printf("%v\n",p2(2))
    }

    func plusTwo() func(int) int { |\longremark{Define a new function that returns a function. %
    See how you you can just write down what you mean;}|
            return func(x int) int { return x + 2 } |\longremark{Function literals at work, %
    we define the +2--function right there in the return statement.}|
    }

    func plusTwo() func(int) int { |\longremark{Define a new function that returns a function. %
    See how you you can just write down what you mean;}|
            return func(x int) int { return x + 2 } |\longremark{Function literals at work, %
    we define the +2--function right there in the return statement.}|
    }


\Question
Here we use a closure:
\begin{lstlisting}
func plusX(x int) func(int) int { |\longremark{Here \citem, we again define a function that returns %
a function;}|
        return func(y int) int { return x + y } |\longremark{We use the \emph{local} variable %
\var{x} in the function literal at \citem.}|
}
\end{lstlisting}
\showremarks
\end{Answer}
